%%%%%%%%%%%%%%%%%%%%%%%%%%%%%%%%%%%%%%%%%%%%%%%%%%%%%%%%%%%%%%%%%%%%%%%%%%%
%                                                                         %
%                                Ricci.tex                                %
%                                                                         %
%              TeX macros for TeX output produced by Ricci.               %
%                                                                         %
%%%%%%%%%%%%%%%%%%%%%%%%%%%%%%%%%%%%%%%%%%%%%%%%%%%%%%%%%%%%%%%%%%%%%%%%%%%
%
% "tensor" macro for typesetting tensors with offset upper & lower
%  indices.
%
% Syntax: \tensor{<symbol>}{\down <sub1> \up <sup1> \down <sub2> ...}
% e.g.
% \tensor R{\down i \up j \down {kl}}
% produces:
%       j
%     R
%      i kl
%
% Ricci uses this macro instead of simply typing R_{i}^j_{kl}, in order to
% force TeX to align the baselines indices of different heights.  This
% problem is particularly noticeable when barred and unbarred indices are
% mixed. 
%
\def\mysavedown#1{\edef\mysubs{\mysubs#1}}
\def\mysaveup#1{\edef\mysups{\mysups#1}}
\def\mydown#1{{\mytensor}_{\vphantom{\mysubs}#1}}
\def\myup#1{{\mytensor}^{\vphantom{\mysups}#1}}
\def\tensor#1#2{
  #1
  \def\mytensor{\vphantom{#1}}
  \def\mysubs{\relax}
  \def\mysups{\relax}
% Make one pass across indices to determine maximum height.
  \let\down=\mysavedown
  \let\up=\mysaveup
  #2
% Now make a second pass to typeset the sub- & superscripts.
  \let\down=\mydown
  \let\up=\myup
  #2
  }

\def\Alt{\mathop{\hbox{Alt}}\nolimits}
\def\Sym{\mathop{\hbox{Sym}}\nolimits}
\def\Del{\mathop\nabla\nolimits}
\def\grad{\mathop{\hbox{grad}}\nolimits}
\def\div{\mathop{\hbox{div}}\nolimits}
\def\extd{\mathop{d}\nolimits}
\def\Lie{\cal L} %% Change this to "\Cal L" for amstex
